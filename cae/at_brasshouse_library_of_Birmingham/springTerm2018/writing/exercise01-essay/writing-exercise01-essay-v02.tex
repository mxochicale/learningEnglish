\documentclass[10pt]{article}
\usepackage[a4paper, total={6in, 10in}]{geometry}

\renewcommand{\abstractname}{Revisions}


\title{Priorities for governments aiming to preserve cultural heritage} 
\author{Miguel P Xochicale\\
Essay 01, Version 03}
\date{ }

\begin{document}
\maketitle
\thispagestyle{empty} %No number



\begin{abstract}
Version 03: \\
- Paragraph one has been rewritten without using "particularly. \\
- Amended word order, avoid word repetition, rephrasing second sentences in paragraph two.
\end{abstract}

One of many challenges for governments is to provide the right amount of funding 
in many areas of which the priorities are generally regarding education, health or 
cultural heritage. 
Specially, for this work, we are mainly interested in discussing which of the 
following aspects can be prioritised by governments. These can be either the increase of funding for museums 
or the pedagogical aspects about the importance of cultural heritage in schools.

Generally, it is believed that education for children is a way to preserve any 
aspect of knowledge for future generations by teaching them the importance of cultural 
heritage as part of their current curriculum. 
Nonetheless, few children are interested in their own culture which is an additional difficulty 
that is not only a responsibility of governments but for families teaching appropriate values.


Similarly, increasing funding for museums is something that goes hand in hand with 
the preservation of cultural heritage where usually new exhibitions require 
substantial investments, for instance, to set up new galleries or invite new artists
to mention but a few. Such is the challenge for small museums where budget is in
general symbolic.

Personally speaking, I do believe that the government should prioritise an increase 
in funding for museums as these are an excellent way to educate not only the children
but also the general public of all ages by making them more aware of the importance of  
their cultural heritage. 
Also, I do think that the increase of funding for museums is an essential investment.
For example, the museum can create its own business model to make itself 
more sustainable for the creation of more interesting
and interactive exhibitions where people can learn more about their own culture.



\end{document}
