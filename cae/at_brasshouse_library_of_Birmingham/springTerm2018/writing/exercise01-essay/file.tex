\documentclass[10pt]{article}
\usepackage[a4paper, total={6in, 10in}]{geometry}



\title{Priorities for governments aiming to preserve cultural heritage} 
\author{Miguel P Xochicale\\
Essay 01, Version 02}
\date{ }

\begin{document}
\maketitle
\thispagestyle{empty} %No number

\begin{abstract}
Version 02: Ammended typos, spelling corrections and reorganising sentences.
\end{abstract}

One particularly challenge for governments is to provide the right amount of funding for 
many priorities regarding education, health or cultural heritage. Particularly,
in this case we are mainly interested in discussing which the following aspects 
can be prioritised by governments. These can be either the increase of funding for museums 
or the pedagogical aspects about the importance of cultural heritage in schools.

Generally, it is believed that education for children is a way to preserve any 
aspect of knowledge for future generations by teaching them the importance of cultural 
heritage as part of their current curriculum. Nonetheless, these can not be entirely 
truth as few children are currently nowhere near interested in their own culture.

Similarly, increasing funding for museums is something that goes hand in hand with 
the preservation of cultural heritage where usually new exhibitions require 
substantial investments, for instance, to set up new galleries or invite new artist
to mention but a few. Such is the challenge for small museums where budget is in
general symbolic.

Personally speaking, I do believe that the government should prioritise an increase 
in funding for museums as these are an excellent way to educate not only the children
but also the general public of all ages by making them more aware of the importance of  
their cultural heritage. 
Also, I do think that the increase of funding for museums is an essential investment
where, for example, the museum can create its own business model to make itself 
more sustainable for the creation of more interesting exhibitions for the public
by creating interactive events where people can learn more about their own culture.



\end{document}
