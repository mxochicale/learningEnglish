\documentclass[11pt]{article}
\usepackage[a4paper, total={6in, 10in}]{geometry}
\renewcommand{\abstractname}{Revisions}




\title{Motivating Children to do regular exercise} 
\author{Miguel P Xochicale \\
Essay02, Version 02}
\date{ }

\begin{document}
\maketitle
\thispagestyle{empty} %No number

\begin{abstract}
Corrections for Version 02:\\
- adding missing verbs, fixing pronouns, articles and prepositions, and diving a long paragraphs into two.
\end{abstract}


Undoubtedly, doing exercise is quite crucial these days where technological advances
are becoming faster and cheaper such as the internet service or the low-cost mobiles phones. 
Such is the worry for parents who certainly want to educate their next generation
not to be only as healthy as possible but also to have good  manners. With that in mind, 
the current debate for parents to motive their children to exercise regularly lies 
between the increase of funding for governments, sport facilities or the improvement 
of parental example.

It is generally believed that if sports facilities were cheaper and mainly accessible, 
more people would utilise them.
Such premise is certainly true to the degree that people would be more persuaded 
to exercise regularly, specially for children who might be more motivated 
to take part in ludic activities that improve their physical health. 
That is one of many reasons for which government is being encouraged to increase 
the funding for building sport centres with accessible prices for people of all ages.

Nonetheless, I personally believe that, parental example is more important to tackle 
the problem from the root rather than the increase of governmental funding. 
Essentially, parents are the role models for their children and certainly for all 
manners and habits of children which are transmitted by example.
Parents, therefore, have a great deal of responsibility to educate their children,
for instance, to eat healthily, to exercise regularly, and to persuade their children 
to live a healthy life.

All in all, it is parents' responsibility to be the role models for their children
to exercise regularly as children in the future will pass better habits and manners 
to their next offspring.





\end{document}
