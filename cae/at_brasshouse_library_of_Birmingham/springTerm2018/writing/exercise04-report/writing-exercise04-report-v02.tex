\documentclass[10pt]{article}
\usepackage{geometry}
 \geometry{
 a4paper,
 total={170mm,257mm},
 left=20mm,
 top=20mm,
bottom=20mm
 }

\usepackage{lineno}
\linenumbers

\usepackage{enumitem}
\setlist{nolistsep}


\renewcommand{\abstractname}{Revisions}

\title{Attitudes to advertising in Mexico} 
\author{Miguel P Xochicale \\
Report 01, Version 02}
\date{14th February 2018}

\begin{document}
\maketitle
\thispagestyle{empty} %No number



\begin{abstract}
Corrections for Version 02: \\
- fixing prepositions, correcting words, amending the articles, fixing verb tenses, fixing word order.\\
- fixing the verb tense of first paragraph of section 2.2.\\
- checking the subject-verb agreement in the sentences. \\
- rephrasing last paragraph.
\end{abstract}


\section{Writing Activity}
\textit{
An international market research company has asked you to write a report
on advertising in your country. The company that has commissioned the 
report wants to know about the most common approaches used and how 
consumers respond to them. You are also asked to suggest changes to 
current approaches or alternative approaches which you believe would
be more effective.
}
\\
Write your report in 220-260 words in an appropriate style.

\section{Report}
\subsection{Introduction}

Undoubtedly, Mexico is a country where marketing strategies of any 
international company can be seriously put to a test, the reason for that
is because of the diversity of both cultural and economical background of 
Mexicans. With that in mind, this report presents a list of three markets 
to give a gist of traditional and common approaches for advertising
in Mexico. 
In addition to the customers' reaction to each of the markets, 
we finalise the report with a list of  recommendations that aim 
to create a more effective impact on people with advertisements
and present a general conclusion.

%Telemarketing? Advertising hoardings? Junk mail? TV advertising? Internet pop-ups? Giveaways in magazines?


\subsection{Types of Marketing in Mexican}
Examples of commercial Mexican markets are many but in this report
we are considering markets that, in my view, have generated a 
major economical gain in the last ten years. 
In each one of the following examples, 
we provide general information about traditional approaches 
for marketing and general responses of customers.
\begin{itemize}
\item Education. The advertisement, specifically, in private universities 
is well made, as investments in this regard are highly relevant for such 
institutions.
For instance, advertisements goes from TV commercials, internet pop-ups,
or even the well known open days where students with their family visit 
the installations.
However, public universities rely on less sophisticated approaches such 
as flyers, spots on the radio and mainly on personal experiences of the ex-graduates.
\item Gastronomy. Advertising in this area is particularly special,
as the Mexican markets have many informal commerce, one of them being gastronomy.
Essentially, what it works better is word of mouth advertising than any of the 
other options. However, for restaurants or big supermarkets
TV advertisings and internet pop ups is the main way to attract people.
\item Tourism. This area is typically advertised by TV commercials
 and customers essentially rely on how well a particular is presented in the video. 
However, customers, specially national ones, usually know someone 
who had been there and then from viva voice, customers 
made their final decision based on those recommendations.
\end{itemize}

\subsection{Recommendations}
Generally speaking, word of mouth advertising is crucial to sell well
in typical Mexican markets and I hardly doubt that changes can be made
in this regard since all related informal advertising are rooted in 
the cultural background of Mexicans. However, I do believe that 
the following emerging technologies can be proposed to improve 
the way different products are advertised in the Mexican market:
\begin{itemize}
\item Virtual reality (VR) can be beneficial for virtual tours
for either a visit to the facilities of an institution
or a view to a tourist resort.
\item Cripto currencies can be a helpful way to attract younger
customers who are more likely to go on a holiday where 
any transaction can be made with the use of digital currencies.
\end{itemize}
Nonetheless, these recommended technologies might not be appropriate
for every Mexican customer.

\subsection{Conclusions}
In my view, I can conclude that today's challenges for traditional
 marketing can be taken as an opportunity so as to use 
the previous mentined emerging technologies and generate conditions 
where win-win situations can be created and 
everyone can be benefited: from a customer, creator of technologies 
to anyone in the marketing companies.
\end{document}
