\documentclass[10pt]{article}
\usepackage[a4paper, top=5mm, bottom=10mm]{geometry}
\renewcommand{\abstractname}{Revisions}


\title{Test3, Writing Part2, Exercise 2} 
\author{Miguel P Xochicale \\
Letter01, Version 02}
\date{14th February 2018}

\begin{document}
\maketitle
\thispagestyle{empty} %No number


\begin{abstract}
Corrections for Version 02: \\
- fixing prepositions, correcting words, adding the right articles\\
- rephrasing second sentence of the second paragraph and last sentence of the last paragraph.
\end{abstract}


\section{Activity}
You receive this letter from a friend who is planning to study abroad:
\textit{
"I'm not sure about going to study abroad anymore. How will I meet people
and find somewhere to live? And I'm worried I won't understand my lessons.
Maybe it's not the right thing for me after all!"}
You decide to write to your friend giving your opinion and offering advice.
Write a letter. 

\section{Letter}


Dear Alex,

It is good to hear from you. I am glad to know that you are still considering
studying abroad after leaving university three years ago.

I can fully understand your current worries in this process as 
I experimented with similar situations four years ago. Well, let me share
various points that you mention in your letter in order to help you to make 
this process less stressful and, I do believe, make it an achievable goal.

Firstly, I personally think that studying abroad is the best experience in my 
lifetime, specially if you consider Western or Asian countries where 
you will totally experience a cultural shock in terms of language, weather, 
social behaviours of the native people and, of course, many other experiences 
that are rarely seen in Central America or South America. 
Another positive aspect of studying abroad is that not only 
you learn to adapt yourself to new environments and conditions 
but also you learn from an interesting personal introspection journey
for which you become a more independent, responsible and decisive person.


With regard to your doubt about meeting people, I think that is essentially 
dependant on your personality and as far as I remember you are quite an outgoing and 
charming person which makes life easier for you in place where you do not know anyone. 
My recommendation for this is that you listen to radio programs  
where native people speak in more realistic environments
as these are very different from 
what you listen to on documentaries or video lectures.

In terms of accommodation, I think, it is the easiest part of this process
of leaving the nest. Usually universities have a special department that help 
students to find an appropriate room. Additionally, there are many agencies 
that you can contact via email or video call. 
There is always space for you as regards of accommodation, 
so you don't have to worry about this one too much.
However, you have to be very precautious about the amount of your available money
for your accommodation as this quantity will vary on the quality of apartment, 
number of people that share the house and the distance from the house to 
university.

In the case of your lessons, I can tell you that lectures --teachers, instructors, professors-- 
usually video record the lectures which makes life easier for you. Additionally, 
the way lectures speak is very clear in order to make the material easily approachable 
and understandable for international students.

Finally, I do think that you have to take a chance and don't let this opportunity pass.
You have nothing to lose. I understand that this situation of uncertainty and going to 
an unknown place in a new country is extremely daunting but you will see after a while
what you consider an stressful period will be the beginning of an
extraordinary adventure both personal and professional.

Regards,
Miguel





\end{document}
