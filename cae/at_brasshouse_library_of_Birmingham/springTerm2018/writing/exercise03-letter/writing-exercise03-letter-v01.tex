\documentclass[10pt]{article}
\usepackage[a4paper, total={6in, 10in}]{geometry}


\title{Test3, Writing Part2, Exercise 2} 
\author{Miguel P Xochicale \\
Letter01, Version 01}
\date{5th February 2018}

\begin{document}
\maketitle
\thispagestyle{empty} %No number


\section{Activity}
You receive this letter from a friend who is planning to study abroad:
\textit{
"I'm not sure about going to study abroad anymore. How will I meet people
and find somewhere to live? And I'm worried I won't understand my lessons.
Maybe it's not the right thing for me after all!"}
You decide to write to your friend giving your opinion and offering advice.
Write a letter. 

\section{Letter}


Dear Friend,

It is good hearing from you. I am glad to know that you are still considering
studying abroad after leaving university three years ago.

I can fully understand your current worries in this process as 
I had experimented similar situations four years ago. Well, let me share
various points that you mention in your letter in order to help you to make 
this process less stressful and, I do believe, make it as an achievable goal.

Firstly, I personally think that studying abroad is the best experience in my 
lifetime, specially if you consider Western or Asian countries where 
you will experiment a real cultural shock in terms of language, weather, 
social behaviours of the native people and, of course, many other experiences 
that are rarely seen in Central America or South America. 
Another positive aspect of studying abroad is the personal experience that you 
gain and certainly many habits, that you learn in the process, making you a 
more independent and decisive person for matter more in your future.

With regard to your doubt about meeting people, I think that is essentially 
dependant on your personality and as far as I remember you are quite outgoing and 
charming person which makes the life easier for you in place where you do not
anyone. My recommendation for this is that you listen as much as you can 
radio programs where native people speak as these are very different from 
what you listen on the documentaries or video lectures.

In terms of accommodation, I think, it is the most easy part of this process
of leaving the nest. Usually universities have a special department that help 
students to find an appropriate room. Additionally, there are many agencies 
that you can contact via email or video call. 
There is always space for you in question of accommodation, 
so you do not have to worry about this one too much.
However, you have to be very precocious about the amount of your available money
for your accommodation as this quantity will vary on the quality of department, 
number of persons that share the house and the distance from the house to 
university.

In the case of your lessons, I can tell you that lectures --teachers, instructors, professors-- 
usually video record the lectures which makes the life easier for you. Additionally, 
the way lectures speak is very clear in order to make the material easily approachable 
and understandable for international students.

Finally, I do think that you have to take a chance and do not let pass this opportunity.
You have nothing to lost. I understand that this situation of uncertainty and going to 
an unknown place in a new country is extremely daunting but you will see after a while
what you see as an stressful period will be the beginning of a fabulous personal and professional
adventure.

Regards,
Miguel





\end{document}
